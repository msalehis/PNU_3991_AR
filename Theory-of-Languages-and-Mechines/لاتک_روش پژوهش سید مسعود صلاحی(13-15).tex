\documentclass{book}
\usepackage{multicol}
\usepackage{xcolor}
\usepackage{graphicx}
\linespread{1.35}
\usepackage{amsmath}
\usepackage{color}
\usepackage{tikz}
\usetikzlibrary{arrows,automata}

\begin{document}

\begin{flushright}
 \texttt{INTRODUCTION} \hspace*{1cm} \textbf{13}
\end{flushright}

\vspace*{0.5cm}
futuristic writings of its inventor, William Gibson, and less so in the more mundane world of network-based education and delivery. Likewise, we do not aggrandize the notion of partnership between networked technology and human beings to indulge in the language of cyborgs or cybernetics--even though we remain open to the notion of the continuous development of some quite astounding technical aids to human processes, many of which will be neurologically linked directly to our bodies. Although much of the context of networking focuses on communication among network uses, we also do not use the term \emph{computer mediated communication.} The Net provides access to data, virtual environments, textbooks, and many other nonhuman reference sources. Describing the use of these resources as communications seems too anthropomorphic for our liking. Thus, we are left with a shortage of precise and well-understood terminology. We have settled on the use of the adjective \emph{net-worked} and the noun Net (with a capital) to describe this context. Net seems to reflect the technical nature of the environment, but also carries with it the context of humaninterconnectedness that is critical to educational applications of the Internet.\\

\hspace*{0.4cm} Our discussion of terminology underscores the multiple functions of the Net. At one level the Net is merely a technology, one that is based on digital transmission, routing, error checking, and sending and receiving of data in many formats. These transmissions may be private and exclusive to as few as two participants or as wide as broadcasts to millions. At the same time, the Net is a rich social environment or con-text in which many aspects of human life, from schooling, to commerce, to sex, are sup-ported. The Net is also a sociological and psychological filter, in which ideas are formatted and in many ways de-contextualized into text or audiovisual constructs. The Net is also a business in which fortunes are made and lost. Finally the Net is a reposi-tory, providing means and tools to store and retrieve a host of cultural, academic, com-mercial, and technical data.\\

 \hspace*{0.4cm} We have experienced even greater difficulty describing non-networked activity, which we often like to contrast to activity mediated via the networked via the network. Describing non-networked research as ''real'' as opposed to ''virtual'' certainly does not work. Describ-ing all aspects of life that are not mediated on a network as ''offline'' activity also seems somewhat condescending and technocentric. We are also not comfortable with the somewhat derogatory reference to humans as ''wetware,'' ''meatware,''or ''liveware.'' Thus, when we are discussing non-networked activity or contexts we usually refer to them as ''face-to-face'' or ''traditional'' and in their educational senes as ''classroom'' or ''campus-based.''\\

\vspace*{0.6cm}
\textbf{---------------------------------------------------------------------------------------------------------------------}\\
\large{
AN e-RESEARCH EXAMPLE
}

\vspace*{0.3cm}
 In the winter of 2001, Liam Rourke and I(Terry Anderson) developed a research proposal ti investigate the capacity and impact of peer moderators in the computer-mediated-communications delivered graduate course that I was teaching. From our own teaching, we were a ware of the excessive time commitments involved in teaching online and of the literature on peer teaching effectiveness. We had developed a tool to assess ''teaching \\
 
 \newpage
 \begin{flushleft}
\textbf{14}\hspace*{1cm} \texttt{CHAPTER ONE}
\end{flushleft}

\vspace*{0.5cm}
presence'' (Anderson, Rourke, Garrison, \& Archer, 2001 that we wanted to apply in a real-life context. We used the Net to scour the ERIC database and Google  (a search engine) to search for related terms like \emph{peer moderators} and \emph{peer teaching,} and we ordered texts not available in our university library using online interlibrary loan request forms. We created a research plan and shared it with a colleague for critical review. We then downloaded  and completed the research ethics forms from our faculty Web page and, of course, submitted them electronically. Upon approval of the project, we drafted a letter of introduction to students, in which we informed them of the intent of the research and the proposed activities. We emailed this letter and opened a forum on a conferencing system for discussion of the research process. In some cases, a follow-up email was required, but eventually all eighteen students gave their consent to participate. We then developed a short Net-based survey on the elements of teaching presence. These results were trian-gulated with information from a transcript analysis. During the six weeks of the experi-ment, we emailed each of the students reminding them of the day they were to complete the weekly online questionnaire. After the course completed, we conducted semi-struc-tured telephone interviews with a sample of the students, applied our transcript analysis instrument with two independent coders, and reflected on our own experiences of the course. From these data sources, we drafted and revised a paper and emailed it to the stu-dents for comments (as a member check). After a final revision, we submitted the paper to the \emph{Journal of Interactive Media in Education} ($http://www.jime.ac.uk$)--a non-blind, peer-revieewed, online journal.

 The article was reviewed by three reviewers and after some minor edits and improvements, it was accepted for publication by the editor. Inaddition, we posted the paper along with additional output from our research group on our own Web dissemination site at $http://www.atl.ualberta.ca.$\\
 \hspace*{0.4cm} Was this an e-research example? Certainly the context and the site of investigation were based on the Net. We used the Net extensively to support data collection and a administration of the project. For example, we conducted our literature review almost exclusively on the Net, used a conference to archive ongoing discussion with students, used email to obtain informed consent and to communicate, developed and administered a Web-based survey, and used the Web in a number of ways for dissemination of results. However, we weren't dogmatically committed to the Net. We used the telephone for interviews, as not all students had IP (Internet Protocol)-based telephony.\\
 In this example, the Web was used in two common applications of e-research. First, it enabled and made more efficient the process of research practice as a means to research and disseminate the results. Second, the Net allowed us to investigate an educational activity taking place on the Net. Rapid communications with subjects throughout the course of the research as well as in vestigation of interaction through transcript analysis shaped the kind and nature of the research process.\\

 \vspace*{0.6cm}
 \large{
\textbf{SUMMARY}
}

\vspace*{0.3cm}
The primary goal of this book is to help the reader understand, appreciate, and con-trol the underlying economics, operating techniques, and ethical considerations of e-research. Research has many characteristics and qualities and operates in many dif-ferent contexts. One of the most important of these qualities is quality itself. Quality research addresses important problems and is honed to find solutions to those prob-lems. It is systematic, transparent, and available to the public. The Net provides us with \\

\newpage

\begin{flushright}
 \texttt{INTRODUCTION} \hspace*{1cm} \textbf{15}
\end{flushright}

\vspace*{0.5cm}
 new tools for quality research as well as the exploration of new fields of knowledge. But the e-researcher must also have a of research skills necessary to conduct quality e-research. These skills are twofold: Internet skills (self-efficacy, mental models, access, terminology, experience, and trouble shooting) and research skills (problem statement, literature review, data collection, data analysis, and dissemination).\\

 \vspace*{0.6cm}
 \textbf{---------------------------------------------------------------------------------------------------------------------}\\
\large{REFERENCES

}

\vspace*{0.1cm}
Anderson, T., Rourke, L., Garrison, R., \& Archer, W. (2001). Assessing teaching presence in com-\\
\hspace*{0.4cm} puter conference transcripts. \emph{Journal of Distance Education, 15(1), 7-23.}\\
Bandura, A. (1977). Self-efficacy: Toward a unifying theory of behavioral change. \emph{psychological Review, \\
\hspace*{0.4cm} 84, 191-215.}\\
Benedikt, M. (1991). Cyberspace: Some proposals. In M. Benedikt (Ed.), \emph{Cyberspace: First steps }\\
\hspace*{0.4cm} (pp. 119-224). Cambridge, MA:MTT Press.\\

 Bradley,J. (2000,January). Online business still needs the basice. \emph{Washington CEO,} [Online]. Avail-\\
\hspace*{0.4cm} able: $http://www.washingtonceo.com/archive/jan00/0100-E-Com.html$.\\
Clarke, R. (2000). \emph{Information wants to be free.} [Online]. Available: http://www.anu.edu.au/people/\\
\hspace*{0.4cm} Roger. Clarke/II/IWtbF.html.\\
Eastin M., \& LaRose, R. (2000). Internet self-efficacy and the psychology of the digital divide. \emph{Jour-\\
\hspace*{0.4cm} nal Of Computer Mediated Communications, 6(1)}. [Online]. Available: http://www.ascusc.org/\\
\hspace*{0.4cm} jcmc/vo16/issue1/eastin.html.\\
Jenkins, S. (2000). \emph{Internet glossary}. [Online]. Available: http://www.unisa.edu.au/itsuhelpdesk/faqs/\\
\hspace*{0.4cm} glossary.htm.\\
Stefik, M. (1996). \emph{Internet dreams: Archetypes, mytbs, and metaphors.} Cambridge, AM:MIT press.\\
Wing, K., Whitehead, P,. \& Maran, R. (1999). \emph{Internet and World Wide Web Simplified} (3rd ed.). New \\
\hspace*{0.4cm}  York: Wiley.\\


\end{document} 
 

